\documentclass[twoside,twocolumn]{article}

\usepackage{blindtext} 
\usepackage{graphicx}
\usepackage[sc]{mathpazo} 
\usepackage[T1]{fontenc} 
\linespread{1.05} 
\usepackage{microtype} 

\usepackage[utf8]{inputenc} 


\usepackage[spanish,english]{babel} 



\usepackage[hmarginratio=1:1,top=32mm,columnsep=20pt]{geometry} 
\usepackage[hang, small,labelfont=bf,up,textfont=it,up]{caption} 
\usepackage{booktabs} 


\usepackage{lettrine} 


\usepackage{enumitem} 
\setlist[itemize]{noitemsep} 


\usepackage{abstract} 
\renewcommand{\abstractnamefont}{\normalfont\bfseries} 
\renewcommand{\abstracttextfont}{\normalfont\small\itshape} 


\usepackage{titlesec} 
\renewcommand\thesection{\Roman{section}} % 
\renewcommand\thesubsection{\roman{subsection}} 
\titleformat{\section}[block]{\large\scshape\centering}{\thesection.}{1em}{} 
\titleformat{\subsection}[block]{\large}{\thesubsection.}{1em}{} 


\usepackage{fancyhdr} 
\pagestyle{fancy} 
\fancyhead{} 
\fancyfoot{} 
\fancyhead[C]{Patrones de Diseño $\bullet$ Octubre 2020 $\bullet$ } 
\fancyfoot[RO,LE]{\thepage} 


\usepackage{titling} 


\usepackage{hyperref} 

\usepackage{listings}
\usepackage{xcolor}

\lstdefinestyle{sharpc}{language=[Sharp]C, frame=lr, rulecolor=\color{blue!80!black}}


%----------------------------------------------------------------------------------------
%	TILULOS
%----------------------------------------------------------------------------------------


\setlength{\droptitle}{-4\baselineskip} 

\pretitle{\begin{center}\Huge\bfseries} 
\posttitle{\end{center}} 
\title{Patrones de Diseño} 
\author{Percy Taquila Carazas, Katerin Merino Quispe, Abraham Lipa Calabilla,
\\Edwart Balcon Coahila, Lisbeth Espinoza Caso}
\date{\today} 
\renewcommand{\maketitlehookd}{

\selectlanguage{english}
\begin{abstract}
\noindent 

\end{abstract}


\selectlanguage{spanish}
\begin{abstract}
\noindent 

\end{abstract}

}

%----------------------------------------------------------------------------------------

\begin{document}

% Print the title
\maketitle

%----------------------------------------------------------------------------------------
%	INTRODUCCION
%----------------------------------------------------------------------------------------

\section{Introduccion}

\lettrine[nindent=0em,lines=3]{L}os



%----------------------------------------------------------------------------------------
%	Desarrollo
%----------------------------------------------------------------------------------------


\section{Desarrollo}

\subsection{Pruebas de API (API testing)}

Las pruebas de API se realizan en la capa más crítica de la aplicación: \textbf{La capa de negocio}, en la que se lleva a cabo la lógica de negocio y las trnasacciones entre la interfaz de usuario y la capa de base de datos ocurren. [1]

\subsubsection{Ventajas}

A continuación se nombran algunas ventajas de las pruebas de API sobre otros tipos de pruebas. [1]

\begin{itemize}
  \item \textbf{Independiente del lenguaje} \\
  Los datos se intercambian a través de formatos XML y JSON, por lo que se puede utilizar cualquier lenguaje para la automatización de pruebas. XML y JSON suelen ser datos estructurados, lo que hace que la verificación sea rápida y estable. También hay bibliotecas integradas para admitir la comparación de datos utilizando estos formatos de datos.
  \item \textbf{Independiente de la GUI} \\
  Las pruebas de API se pueden realizar en la aplicación antes de las pruebas de GUI. Las pruebas tempranas significan retroalimentación temprana y una mejor productividad del equipo. Las funcionalidades centrales de la aplicación se pueden probar para exponer pequeños errores y evaluar las fortalezas de la compilación.
  \item \textbf{Cobertura de prueba mejorada} \\
  La mayoría de los servicios web / API tienen especificaciones, lo que le permite crear pruebas automatizadas con una alta cobertura, incluidas las pruebas funcionales y las no funcionales.
  \item \textbf{Lanzamientos más rápidos} \\
  Es común que la ejecución de pruebas de API ahorre hasta ocho horas en comparación con las pruebas de IU, lo que permite a los equipos de desarrollo de software lanzar productos más rápido.
\end{itemize}




\subsection{Pruebas de API con Katalon Studio}

\subsubsection{Katalon Studio}

Katalon Studio es una herramienta de licencia gratuita lanzada en enero de 2015 con un motor basado en Selenium. Principalmente, Katalon está diseñado para crear y reutilizar scripts de prueba automatizados para UI sin codificación. Katalon Studio permite ejecutar pruebas automatizadas de elementos de la interfaz de usuario, incluidas ventanas emergentes, iFrames y tiempo de espera. La herramienta se puede ejecutar en Microsoft Windows, macOS y Linux. [2]

\subsubsection{Compañias que usan Katalon}

A continuación se nombran algunas de las compañias que usan Katalon. [3]

\begin{itemize}
  \item New Jersey Institute of Technology
  \item Metric Tree Labs
  \item BAE Systems
  \item Cognitio Corp.
  \item EZLynx
\end{itemize}

\subsubsection{Características}

A continuación se mencionan algunas características de las pruebas de API realizadas con Katalon. [1]

\begin{itemize}
  \item \textbf{IDE productivo para la automatización de API} \\
  Optimiza los procesos de secuencias de comandos, depuración y mantenimiento de pruebas con autocompletado, inspección de código, fragmentos, referencias rápidas, depurador, interfaz dual, etc.  
  \item \textbf{Centrado en API} \\
  Katalon admite todo tipo de solicitudes REST, SOAP / 1.1 y SOAP / 1.2. Sus pruebas se pueden importar desde Swagger, Postman y WSDL.
  \item \textbf{Construido para trabajar con marcos modernos} \\
  Pruebas basadas en datos simplificadas con múltiples fuentes de datos (por ejemplo, XLS, CSV) y bases de datos compatibles. Katalon también admite BDD con archivos Cucumber y editor nativo Gherkin.
  \item \textbf{Plataforma sostenible y escalable} \\
  Habilita las prácticas de Integración Continua y DevOps. Fácil implementación con contenedores Docker. Ejecución local y remota con analítica en tiempo real.
  \item \textbf{Mantenimiento mínimo} \\
  Reutilice artefactos de prueba en diferentes proyectos. Defina escenarios de prueba y planes de ejecución con capacidades de gestión de conjuntos de pruebas.
\end{itemize}

\subsection{Pruebas de API con Postman}

\subsubsection{Postman}

Postman es una plataforma de colaboración para el desarrollo de APIs. Las funciones de Postman simplifican cada paso de la creación de una API y agilizan la colaboración para que pueda crear mejores APIs, más rápido. [5]

\subsubsection{Características}

A continuación se mencionan algunas características de Postman. [5]

\begin{itemize}
  \item \textbf{Es gratis y fácil de comenzar} \\
  Simplemente descargue la aplicación Postman y envíe su primera solicitud en minutos. Postman se puede descargar y utilizar de forma gratuita para equipos de cualquier tamaño.

  \item \textbf{Amplio soporte para todas las API y esquemas} \\
  Realice cualquier tipo de llamada a la API (REST, SOAP o HTTP simple) e inspeccione fácilmente incluso las respuestas más grandes. Postman también tiene soporte integrado para formatos de datos populares como OpenAPI GraphQL y RAML.

  \item \textbf{Es extensible} \\
  Personalice Postman según sus necesidades con la API Postman. Integre conjuntos de pruebas en su servicio de CI / CD preferido con Newman, nuestro corredor de recopilación de línea de comandos.

  \item \textbf{Soporte y comunidad} \\
  Continuamente realizamos mejoras y agregamos nuevas funciones en función de los comentarios de nuestra comunidad de más de 13 millones de usuarios, que también pueden ayudarlo a aprovechar al máximo Postman en nuestro foro comunitario.
\end{itemize}

\subsubsection{Compañias que usan Postman}

A continuación se nombran algunas de las compañias que usan Postman. [4]

\begin{itemize}
  \item Fiserv
  \item Caterpillar Inc.
  \item Wells Fargo
  \item U.S. Bank
  \item CoStar Group
\end{itemize}



%----------------------------------------------------------------------------------------
%	Conclusiones
%----------------------------------------------------------------------------------------


\section{Conclusiones}

La conclusión 
%----------------------------------------------------------------------------------------
%	Recomendaciones
%----------------------------------------------------------------------------------------

\section{Recomendaciones}


\begin{itemize}
\item Cuando se conoce el efecto colateral que conlleva el patrón de diseño y es viable la aparición de este efecto.

\end{itemize}



%----------------------------------------------------------------------------------------
%	BIBLIOGRAFIA
%----------------------------------------------------------------------------------------

\selectlanguage{spanish}
\begin{thebibliography}{99} 

\bibitem[1]{}
\newblock Katalon LLC. (2019, 22 octubre). What is API Testing? | Definition, Benefits, Types \& Tool. Katalon Solution. https://www.katalon.com/api-testing/
\bibitem[2]{}
\newblock Editor. (2019, 9 diciembre). The Good and the Bad of Katalon Studio Automation Testing Tool. AltexSoft. https://www.altexsoft.com/blog/engineering/the-good-and-the-bad-of-katalon-studio-automation-testing-tool/
\bibitem[3]{}
\newblock HG Insights. (s. f.). Companies Using Katalon Studio, Market Share, Customers and Competitors. Recuperado 4 de diciembre de 2020, de https://discovery.hgdata.com/product/katalon-studio
\bibitem[4]{}
\newblock HG Insights. (s. f.-b). Companies Using Postman, Market Share, Customers and Competitors. Recuperado 4 de diciembre de 2020, de https://discovery.hgdata.com/product/postman
\bibitem[5]{}
\newblock Postman. (s. f.). Postman | The Collaboration Platform for API Development. Recuperado 4 de diciembre de 2020, de https://www.postman.com/
\end{thebibliography}


%----------------------------------------------------------------------------------------


\end{document}
